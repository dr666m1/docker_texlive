\documentclass{jlreq}
%math
\usepackage{mathtools,amssymb,bm}
%image
\usepackage{graphicx,xcolor}
%box
\usepackage{tcolorbox}
\tcbuselibrary{skins,breakable}
%table, figure
\usepackage{float}
\begin{document}
\section{概要}
texファイルをコンパイルするためのdockerイメージです。

\section{使い方}
\subsection{基本}
以下のコマンドを実行すると、
カレントディレクトリのtexファイルからpdfを作成します。
複数のtexファイルが見つかった場合、
全てのファイルをコンパイルします。

\begin{tcolorbox}[breakable]
\setlength{\baselineskip}{12pt}
\begin{verbatim}
docker container run \
  --rm -it \
  -v $(pwd):/home/app/sync \
  ghcr.io/dr666m1/texlive
\end{verbatim}

\end{tcolorbox}

\subsection{オプション}
\subsubsection{o}
outputの略です。ただし
\verb|--output|ではなく
\verb|-o|と入力してください。

\begin{description}
  \item[all] 全てのファイルを出力
  \item[png] pdfをpngに変換し、pngのみ出力
\end{description}

\subsubsection{ヘルプ}
末尾に\verb|--help|を付けるとWebサーバーが起動し、
ブラウザからこのpdfを起動できます。
その際は
\verb|-p 8080:8080|を指定してください。
\end{document}
